\documentclass{book}
\usepackage{amsmath}
\usepackage{amssymb}

\title{Introduction to Calculus}
\author{Test Author}
\date{\today}

\begin{document}

\maketitle

\chapter{Limits and Continuity}

\section{Understanding Limits}

A limit describes the value that a function approaches as the input approaches some value. We write this as:

\[ \lim_{x \to a} f(x) = L \]

This means that as $x$ gets closer and closer to $a$, the function $f(x)$ gets closer and closer to $L$.

\subsection{Example: Simple Limit}

Consider the function $f(x) = 2x + 1$. What is $\lim_{x \to 3} f(x)$?

As $x$ approaches 3, we can substitute:
\[ \lim_{x \to 3} (2x + 1) = 2(3) + 1 = 7 \]

\section{Continuity}

A function is continuous at a point $x = a$ if:
\begin{enumerate}
    \item $f(a)$ is defined
    \item $\lim_{x \to a} f(x)$ exists
    \item $\lim_{x \to a} f(x) = f(a)$
\end{enumerate}

\chapter{Derivatives}

\section{Definition of the Derivative}

The derivative of a function $f(x)$ at a point $x$ is defined as:

\[ f'(x) = \lim_{h \to 0} \frac{f(x+h) - f(x)}{h} \]

This represents the instantaneous rate of change of the function at that point.

\subsection{Power Rule}

For any function of the form $f(x) = x^n$ where $n$ is a constant:

\[ \frac{d}{dx}(x^n) = nx^{n-1} \]

\textbf{Example:} If $f(x) = x^3$, then $f'(x) = 3x^2$.

\section{Common Derivatives}

Here are some important derivative formulas:

\begin{align}
\frac{d}{dx}(c) &= 0 \quad \text{(constant rule)} \\
\frac{d}{dx}(x) &= 1 \\
\frac{d}{dx}(\sin x) &= \cos x \\
\frac{d}{dx}(\cos x) &= -\sin x \\
\frac{d}{dx}(e^x) &= e^x \\
\frac{d}{dx}(\ln x) &= \frac{1}{x}
\end{align}

\chapter{Integration}

\section{Antiderivatives}

An antiderivative of a function $f(x)$ is a function $F(x)$ such that $F'(x) = f(x)$.

The indefinite integral is written as:
\[ \int f(x) \, dx = F(x) + C \]

where $C$ is the constant of integration.

\section{Definite Integrals}

The definite integral from $a$ to $b$ is:

\[ \int_a^b f(x) \, dx = F(b) - F(a) \]

This represents the area under the curve $f(x)$ from $x = a$ to $x = b$.

\subsection{Example: Computing Area}

Find the area under $f(x) = x^2$ from $x = 0$ to $x = 1$:

\begin{align}
\int_0^1 x^2 \, dx &= \left[ \frac{x^3}{3} \right]_0^1 \\
&= \frac{1^3}{3} - \frac{0^3}{3} \\
&= \frac{1}{3}
\end{align}

\section{Fundamental Theorem of Calculus}

The Fundamental Theorem of Calculus connects differentiation and integration:

If $F(x) = \int_a^x f(t) \, dt$, then $F'(x) = f(x)$.

This powerful theorem tells us that differentiation and integration are inverse operations.

\end{document}
